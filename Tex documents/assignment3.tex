\let\negmedspace\undefined{}
\let\negthickspace\undefined{}
%\RequirePackage{amsmath}
\documentclass[journal,12pt,twocolumn]{IEEEtran}
%
% \usepackage{setspace}
 \usepackage{gensymb}
%\doublespacing
 \usepackage{polynom}
%\singlespacing
%\usepackage{silence}
%Disable all warnings issued by latex starting with "You have..."
%\usepackage{graphicx}
\usepackage{amssymb}
%\usepackage{relsize}
\usepackage[cmex10]{amsmath}
%\usepackage{amsthm}
%\interdisplaylinepenalty=2500
%\savesymbol{iint}
%\usepackage{txfonts}
%\restoresymbol{TXF}{iint}
%\usepackage{wasysym}
\usepackage{amsthm}
%\usepackage{pifont}
%\usepackage{iithtlc}
% \usepackage{mathrsfs}
% \usepackage{txfonts}
 \usepackage{stfloats}
% \usepackage{steinmetz}
 \usepackage{bm}
% \usepackage{cite}
% \usepackage{cases}
% \usepackage{subfig}
%\usepackage{xtab}
\usepackage{longtable}
%\usepackage{multirow}
%\usepackage{algorithm}
%\usepackage{algpseudocode}
\usepackage{enumitem}
 \usepackage{mathtools}
 \usepackage{tikz}
% \usepackage{circuitikz}
% \usepackage{verbatim}
%\usepackage{tfrupee}
\usepackage[breaklinks=true]{hyperref}
%\usepackage{stmaryrd}
%\usepackage{tkz-euclide} % loads  TikZ and tkz-base
%\usetkzobj{all}
\usepackage{listings}
    \usepackage{color}                                            %%
    \usepackage{array}                                            %%
    \usepackage{longtable}                                        %%
    \usepackage{calc}                                             %%
    \usepackage{multirow}                                         %%
    \usepackage{hhline}                                           %%
    \usepackage{ifthen}                                           %%
  %optionally (for landscape tables embedded in another document): %%
    \usepackage{lscape}     
% \usepackage{multicol}
% \usepackage{chngcntr}
%\usepackage{enumerate}
%\usepackage{wasysym}
%\newcounter{MYtempeqncnt}
\DeclareMathOperator*{\Res}{Res}
\DeclareMathOperator*{\equals}{=}
%\renewcommand{\baselinestretch}{2}
\renewcommand\thesection{\arabic{section}}
\renewcommand\thesubsection{\thesection.\arabic{subsection}}
\renewcommand\thesubsubsection{\thesubsection.\arabic{subsubsection}}
\renewcommand\thesectiondis{\arabic{section}}
\renewcommand\thesubsectiondis{\thesectiondis.\arabic{subsection}}
\renewcommand\thesubsubsectiondis{\thesubsectiondis.\arabic{subsubsection}}
% correct bad hyphenation here
\hyphenation{op-tical net-works semi-conduc-tor}
\def\inputGnumericTable{}                                 %%
\lstset{
%language=C,
frame=single, 
breaklines=true,
columns=fullflexible
}
%\lstset{
%language=tex,
%frame=single, 
%breaklines=true
%}

\title{Assignment 1 \\ \Large AI1110: Probability and Random Variables \\ \large Indian Institute of Technology Hyderabad}
\author{Aayush Prabhu \\ \normalsize AI21BTECH11002 \\ \vspace*{20pt} \normalsize  19 April 2022 \\ \vspace*{20pt} \Large CBSE Probabaility Grade 10}
\begin{document}
       \maketitle
       \textbf{Question 20}\\
       \textbf{Question:} Suppose you drop a die at random in the rectangular region as shown in the figure. what is the probability that the die will land in the circle with diameter 1m? 
       \begin{figure}[h]
       \includegraphics[width=0.5\textwidth]{Figure_1.png}
       \caption{Rectangular region with length $3m$ and breadth $2m$, and a circle with diameter with $1m $ in it.}
       \label{fig:my_label}
       \end{figure}
       \textbf{Solution:}\\
       \begin{table}[h]
    \label{tab:table1}
    \begin{tabular}{l|c|c} 
    \hline
      \textbf{Parameter} & \textbf{Symbol} & \textbf{Value}\\
      \hline
      Radius Of Circle (in metre)  & $r$ & $0.5$ \\
      \hline
      Length of rectangle(in metre) & $l$ & $3$\\
      \hline
      Breadth of Rectangle(in metre) & $b$ & $2$\\
      \hline
      Area of Circle & $A_c$, where $A_c=\pi r^2$ & To be Calculated\\
      \hline
      Area of Rectangle & $A_r$, where $A_r=l\times b$ & To be Calculated\\
      \hline
    \end{tabular}
    \end{table}\\
           Probability of die falling in the circle would be:\\
       \begin{align}
       P=\dfrac{Favourable Area}{Total Area}\\
       \therefore P=\dfrac{Area of Circle}{Area of Rectangle}\\
       \end{align}
    Area of Circle $A_c=\pi\times(0.5)^2 \implies A_c=0.785$\\
    Area of Rectangle $A_r=l\times b \implies A_r=6$\\
    Hence Probability $P=\dfrac{0.785}{6} \implies P=0.1308$\\ 
\end{document}