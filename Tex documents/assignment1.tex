\documentclass[journal,12pt,twocolumn]{IEEEtran}

\usepackage{tfrupee}
\usepackage{enumitem}
\usepackage{amsmath}
\usepackage{amssymb}


\title{Assignment 1 \\ \Large AI1110: Probability and Random Variables \\ \large Indian Institute of Technology Hyderabad}
\author{Aayush Prabhu \\ \normalsize AI21BTECH11002 \\ \vspace*{20pt} \normalsize  2 April 2022 \\ \vspace*{20pt} \Large ICSE 2014 Grade 10}
\begin{document}
       \maketitle
       \textbf{Question 2(b)}\\
       \textbf{Question:} Shahrukh opened a Recurring Deposit Account in a bank and deposited \rupee$800$ per month for $1.5$ years. If he received \rupee$15,084$ at the time of maturity, find the rate of interest per annum.\\
       \textbf{Solution:} Let us first bring up the general case.\\
       Let,\\
       a) Deposit per month be '$d$'.\\
       b) Time(in months) be '$t$'.\\
       c) Total received amount be '$r$'.\\
       d) Total interest be '$i$'.\\
       e) Rate of interest '$I$'.\\
       Total Interest would be the Total amount received minus the Total amount deposited, also Rate of Interest per annum would be Total Interest divided by time in years.\\\\
       $\therefore$ $i=r-dt$ and $I=\dfrac{12i}{t}$\\\\
       Now the values given in the question are:\\
       $d=$ \rupee$800$, $t=18$(as $1.5$ years is equal to $1.5\times12$ months, i.e $18$ months), $r=$ \rupee$15,084$.\\
       Arranging all these values in a table
       \begin{table}[h!]
    \label{tab:table1}
    \begin{tabular}{l|c|c} 
    \hline
      \textbf{Parameter} & \textbf{Symbol} & \textbf{Value}\\
      \hline
      Deposit per month & $d$ & \rupee$800$ \\
      \hline
      Time(in months) & $t$ & $18$\\
      \hline
      Total received amount & $r$ & \rupee$15,084$\\
      \hline
      Total Interest & $i$, where $i=r-dt$ & To be Calculated\\
      \hline
      Rate of Interest & $I$, where $I=\dfrac{12i}{t}$ & To be Calculated\\
      \hline
    \end{tabular}
\end{table}
       \\$\therefore$ $i=$ \rupee$15,084$ $-$\rupee$(800\times18)$\\
       $\Rightarrow$ $i=$ \rupee$15,084$ $-$\rupee$14,400$\\
       $\Rightarrow$ $i=$ \rupee$684$\\\\
       Now $I=\dfrac{12i}{t}$
       $\Rightarrow$ $I=\dfrac{12\times684}{18}$\\\\
       $\therefore$ $I=$ \rupee$456$\\
       $\therefore$ Rate of interest per annum is \rupee$456$.
\end{document}