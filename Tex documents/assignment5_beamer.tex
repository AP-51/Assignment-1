\let\negmedspace\undefined{}
\let\negthickspace\undefined{}
\documentclass{beamer}
\usetheme{CambridgeUS}
 \usepackage{gensymb}
 \usepackage{polynom}
\usepackage{amssymb}
%\usepackage[cmex10]{amsmath}
\usepackage{amsthm}
 \usepackage{stfloats}
\usepackage{bm}
 \usepackage{longtable}
 \usepackage{enumitem}
 \usepackage{mathtools}
 \usepackage{tikz}
 %\usepackage[breaklinks=true]{hyperref}
\usepackage{listings}
\usepackage{color}                                            
\usepackage{array}                                            
\usepackage{longtable}                                        
\usepackage{calc}                                             
    \usepackage{multirow}                                         
    \usepackage{hhline}                                           
    \usepackage{ifthen}                                           
    \usepackage{lscape}     
\DeclareMathOperator*{\Res}{Res}
\DeclareMathOperator*{\equals}{=}
\renewcommand\thesection{\arabic{section}}
\renewcommand\thesubsection{\thesection.\arabic{subsection}}
\renewcommand\thesubsubsection{\thesubsection.\arabic{subsubsection}}
% \renewcommand\thesectiondis{\arabic{section}}
% \renewcommand\thesubsectiondis{\thesectiondis.\arabic{subsection}}
% \renewcommand\thesubsubsectiondis{\thesubsectiondis.\arabic{subsubsection}}
\hyphenation{op-tical net-works semi-conduc-tor}
\def\inputGnumericTable{}                                 %%
\lstset{
frame=single, 
breaklines=true,
columns=fullflexible
}
\begin{document}
% \newtheorem{theorem}{Theorem}[section]
% \newtheorem{problem}{Problem}
% \newtheorem{proposition}{Proposition}[section]
% \newtheorem{lemma}{Lemma}[section]
% \newtheorem{corollary}[theorem]{Corollary}
% \newtheorem{example}{Example}[section]
% \newtheorem{definition}[problem]{Definition}
\newcommand{\BEQA}{\begin{eqnarray}}
\newcommand{\EEQA}{\end{eqnarray}}
\newcommand{\define}{\stackrel{\triangle}{=}}
\newcommand*\circled[1]{\tikz[baseline= (char.base)]{
    \node[shape=circle,draw,inner sep=2pt] (char) {#1};}}
\bibliographystyle{IEEEtran}
\providecommand{\mbf}{\mathbf}
\providecommand{\pr}[1]{\ensuremath{\Pr\left(#1\right)}}
\providecommand{\qfunc}[1]{\ensuremath{Q\left(#1\right)}}
\providecommand{\sbrak}[1]{\ensuremath{{}\left[#1\right]}}
\providecommand{\lsbrak}[1]{\ensuremath{{}\left[#1\right.]}}
\providecommand{\rsbrak}[1]{\ensuremath{{}\left[#1\right.]}}
\providecommand{\brak}[1]{\ensuremath{\left(#1\right)}}
\providecommand{\lbrak}[1]{\ensuremath{\left(#1\right.)}
\providecommand{\rbrak}[1]{\ensuremath{\left[#1\right.]}}}
\providecommand{\cbrak}[1]{\ensuremath{\left\{#1\right\}}}
\providecommand{\lcbrak}[1]{\ensuremath{\left\{#1\right.}}
\providecommand{\rcbrak}[1]{\ensuremath{\left.#1\right\}}}
\theoremstyle{remark}
\newtheorem{rem}{Remark}
\newcommand{\sgn}{\mathop{\mathrm{sgn}}}
\providecommand{\abs}[1]{\left\vert#1\right\vert}
\providecommand{\res}[1]{\Res\displaylimits_{#1}} 
\providecommand{\norm}[1]{\left\lVert#1\right\rVert}
\providecommand{\mtx}[1]{\mathbf{#1}}
\providecommand{\mean}[1]{E\left[ #1 \right]}
\providecommand{\fourier}{\overset{\mathcal{F}}{ \rightleftharpoons}}
\providecommand{\system}{\overset{\mathcal{H}}{ \longleftrightarrow}}
% \newcommand{\solution}{\noindent \textbf{Solution: }}
\newcommand{\cosec}{\,\text{cosec}\,}
\newcommand*{\permcomb}[4][0mu]{{{}^{#3}\mkern#1#2_{#4}}}
\newcommand*{\perm}[1][-3mu]{\permcomb[#1]{P}}
\newcommand*{\comb}[1][-1mu]{\permcomb[#1]{C}}
\renewcommand{\thetable}{\arabic{table}} 
\providecommand{\dec}[2]{\ensuremath{\overset{#1}{\underset{#2}{\gtrless}}}}
\newcommand{\myvec}[1]{\ensuremath{\begin{pmatrix}#1\end{pmatrix}}}
\newcommand{\mydet}[1]{\ensuremath{\begin{vmatrix}#1\end{vmatrix}}}
\numberwithin{equation}{section}
\numberwithin{figure}{section}
\numberwithin{table}{section}
\makeatletter
\@addtoreset{figure}{problem}
\makeatother
\let\StandardTheFigure\thefigure{}
\let\vec\mathbf{}
%\renewcommand{\thefigure}{\theproblem}
\title{Assignment 5 Papoulis-Pillai Chapter 2 Example 2.15 }
\author{Aayush Prabhu (AI21BTECH11002)}
\date{\today}
\logo{\large \LaTeX}

\begin{frame}
    \titlepage 
  \end{frame}
  \logo{}
  \begin{frame}{Outline}
    \tableofcontents
  \end{frame}
       \section{Question}
       \begin{frame}{Question}
       \textbf{Question:}A certain test for a particular cancer is known to be 95 percent accurate. A person submits to the test and the results are positive. Suppose that the person comes from a population
of 100,000, where 2000 people suffer from that disease. What can we conclude about the probability that the person under test has that particular cancer? .\\
        \end{frame}
        \section{Solution}
        \begin{frame}{Solution}
       \textbf{Solution:}The test is known to be 95 percent accurate, which means that 95 percent of all positive tests are correct and 95 percent of all negative tests are correct.\\
        Let the event $T>0$ stand for the test being positive and $T<0$ stand for the test being negative. Also let $H$ and $C$ represent the sets of healthy people and cancerous patients respectively.\\
        \begin{align}
        &\therefore P(T>0|C)= 0.95 && P(T>0|H)= 0.05\\
        &P(T<0|C)= 0.05 && P(T<0|H)= 0.95\\
        \end{align}
        But in this case we have 98000 healthy people and 2000 sick people $\therefore$ without any other conditions $P(H)=0.98$  and $P(C)=0.08$. 
        \end{frame}
        \begin{frame}
        Using Bayes' theorem, probability that the person suffers from cancer givent that the test result is positive:
        \begin{align}
        & P(C|T>0)= \dfrac{P(T>0|C)P(C)}{P(T>0)}= \dfrac{P(T>0|C)P(C)}{P(T>0|C)P(C)+P(T>0|H)P(H)}\nonumber \\
        & = \dfrac{0.95\times0.02}{0.95\times0.02+0.05\times0.98} &&=0.279 \nonumber
        \end{align}
        This result of this test states that if the test is taken by a person from this population without knowing that they have the disease or not, then a positive test states that the person has a 27.9 percent chance of having the disease. However if the person knows that they have the disease then probability of getting a positive test result i.e $P(T>0|C)=0.95$ or 95 percent \\
        \end{frame}
\end{document}