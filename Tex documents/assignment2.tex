\let\negmedspace\undefined{}
\let\negthickspace\undefined{}
%\RequirePackage{amsmath}
\documentclass[journal,12pt,twocolumn]{IEEEtran}
%
% \usepackage{setspace}
 \usepackage{gensymb}
%\doublespacing
 \usepackage{polynom}
%\singlespacing
%\usepackage{silence}
%Disable all warnings issued by latex starting with "You have..."
%\usepackage{graphicx}
\usepackage{amssymb}
%\usepackage{relsize}
\usepackage[cmex10]{amsmath}
%\usepackage{amsthm}
%\interdisplaylinepenalty=2500
%\savesymbol{iint}
%\usepackage{txfonts}
%\restoresymbol{TXF}{iint}
%\usepackage{wasysym}
\usepackage{amsthm}
%\usepackage{pifont}
%\usepackage{iithtlc}
% \usepackage{mathrsfs}
% \usepackage{txfonts}
 \usepackage{stfloats}
% \usepackage{steinmetz}
 \usepackage{bm}
% \usepackage{cite}
% \usepackage{cases}
% \usepackage{subfig}
%\usepackage{xtab}
\usepackage{longtable}
%\usepackage{multirow}
%\usepackage{algorithm}
%\usepackage{algpseudocode}
\usepackage{enumitem}
 \usepackage{mathtools}
 \usepackage{tikz}
% \usepackage{circuitikz}
% \usepackage{verbatim}
%\usepackage{tfrupee}
\usepackage[breaklinks=true]{hyperref}
%\usepackage{stmaryrd}
%\usepackage{tkz-euclide} % loads  TikZ and tkz-base
%\usetkzobj{all}
\usepackage{listings}
    \usepackage{color}                                            %%
    \usepackage{array}                                            %%
    \usepackage{longtable}                                        %%
    \usepackage{calc}                                             %%
    \usepackage{multirow}                                         %%
    \usepackage{hhline}                                           %%
    \usepackage{ifthen}                                           %%
  %optionally (for landscape tables embedded in another document): %%
    \usepackage{lscape}     
% \usepackage{multicol}
% \usepackage{chngcntr}
%\usepackage{enumerate}
%\usepackage{wasysym}
%\newcounter{MYtempeqncnt}
\DeclareMathOperator*{\Res}{Res}
\DeclareMathOperator*{\equals}{=}
%\renewcommand{\baselinestretch}{2}
\renewcommand\thesection{\arabic{section}}
\renewcommand\thesubsection{\thesection.\arabic{subsection}}
\renewcommand\thesubsubsection{\thesubsection.\arabic{subsubsection}}
\renewcommand\thesectiondis{\arabic{section}}
\renewcommand\thesubsectiondis{\thesectiondis.\arabic{subsection}}
\renewcommand\thesubsubsectiondis{\thesubsectiondis.\arabic{subsubsection}}
% correct bad hyphenation here
\hyphenation{op-tical net-works semi-conduc-tor}
\def\inputGnumericTable{}                                 %%
\lstset{
%language=C,
frame=single, 
breaklines=true,
columns=fullflexible
}
%\lstset{
%language=tex,
%frame=single, 
%breaklines=true
%}

\title{Assignment 1 \\ \Large AI1110: Probability and Random Variables \\ \large Indian Institute of Technology Hyderabad}
\author{Aayush Prabhu \\ \normalsize AI21BTECH11002 \\ \vspace*{20pt} \normalsize  19 April 2022 \\ \vspace*{20pt} \Large ICSE 2017 Grade 12}
\begin{document}
       \newcommand{\myvec}[1]{\ensuremath{\begin{pmatrix}#1\end{pmatrix}}}
       \let\vec\mathbf
       \maketitle
       \textbf{Question 2(b)}\\
       \textbf{Question:} Given that: \begin{align} \vec{A} = \myvec{1 & -1 & 0\\2 & 3 & 4\\0 & 1 & 2}\nonumber \\ and\:\vec{B} = \myvec{2 & 2 & -4\\-4 & 2 & -4\\2 & -1 &5} \nonumber 
       \end{align}, find $\vec{AB}$.\\
       Using this result, solve the following system of equation:
       \begin{align} x-y=3,\:2x+3y+4z=17\:and\:y+2z=7\nonumber 
       \end{align}\\
       \textbf{Solution:} 
       \begin{align}
          &\vec{AB} = \myvec{1 & -1 & 0\\2 & 3 & 4\\0 & 1 & 2}.\myvec{2 & 2 & -4\\-4 & 2 & -4\\2 & -1 &5}\nonumber&&\\
        &\vec{AB} = \myvec{2+4+0 & 2-2+0 & -4+4+0\\4-12+8 & 4+6-4 & -8-12+20\\0-4+4 & 0+2-2 & 0-4+10}\nonumber&&\\
        &\implies \vec{AB} = \myvec{6 & 0 & 0\\0 & 6 & 0\\0 & 0 & 6}&&\\
        &\implies \vec{AB} = 6\vec{I}&&
        \end{align} 
        Where $\vec{I}$ is the Identity Matrix of order 3\\
        Now, let us write the following system of equations in matrix form:
        \begin{align} 
        &x-y=3,\:2x+3y+4z=17\: and \:y+2z=7 &&\nonumber\\
        &\implies \myvec{1 & -1 & 0\\2 & 3 & 4\\0 & 1 & 2}.\myvec{x\\y\\z} = \myvec{3\\17\\7}&& \nonumber
        \end{align} 
        Now $\myvec{1 & -1 & 0\\2 & 3 & 4\\0 & 1 & 2}$ is basically $\vec{A}$.\\
        Let $\myvec{x\\y\\z}$ be $\vec{C}$ and $\myvec{3\\17\\7}$ be $\vec{D}$
        $ \therefore \vec{AC}=\vec{D}$
        To solve this system of equations we need to find $\vec{C}$\\
        \begin{align}
        &\implies \vec{C} = \vec{A}^{-1}\vec{D} &&\\
        &and, \vec{AB}=6\vec{I} &&\\
        &\implies \vec{A}^{-1} = \dfrac{\vec{B}}{6} &&\\
        &\implies \vec{C} = \dfrac{\vec{BD}}{6 }&&\\
        &\therefore \vec{C} = \dfrac{1}{6}\myvec{2 & 2 & -4\\-4 & 2 &-4\\2 & -1 & 5}.\myvec{3\\17\\7} &&\\
        &\implies \vec{C} = \dfrac{1}{6}\myvec{6+34-28\\-12+34-28\\6-17+35 }&&\\
        &\implies \vec{C} = \dfrac{1}{6}\myvec{12\\-6\\24} &&\\
        &\implies \vec{C} = \myvec{2\\-1\\4} &&\\
        &\therefore x=2,y=-1,z=4&&
        \end{align} is the solution for this system of equations.\\
\end{document}