\let\negmedspace\undefined{}
\let\negthickspace\undefined{}
\documentclass{beamer}
\usetheme{CambridgeUS}
 \usepackage{gensymb}
 \usepackage{polynom}
\usepackage{amssymb}
%\usepackage[cmex10]{amsmath}
\usepackage{amsthm}
 \usepackage{stfloats}
\usepackage{bm}
 \usepackage{longtable}
 \usepackage{enumitem}
 \usepackage{mathtools}
 \usepackage{tikz}
 %\usepackage[breaklinks=true]{hyperref}
\usepackage{listings}
\usepackage{color}                                            
\usepackage{array}                                            
\usepackage{longtable}                                        
\usepackage{calc}                                             
    \usepackage{multirow}                                         
    \usepackage{hhline}                                           
    \usepackage{ifthen}                                           
    \usepackage{lscape}     
\DeclareMathOperator*{\Res}{Res}
\DeclareMathOperator*{\equals}{=}
\renewcommand\thesection{\arabic{section}}
\renewcommand\thesubsection{\thesection.\arabic{subsection}}
\renewcommand\thesubsubsection{\thesubsection.\arabic{subsubsection}}
% \renewcommand\thesectiondis{\arabic{section}}
% \renewcommand\thesubsectiondis{\thesectiondis.\arabic{subsection}}
% \renewcommand\thesubsubsectiondis{\thesubsectiondis.\arabic{subsubsection}}
\hyphenation{op-tical net-works semi-conduc-tor}
\def\inputGnumericTable{}                                 %%
\lstset{
frame=single, 
breaklines=true,
columns=fullflexible
}
\begin{document}
% \newtheorem{theorem}{Theorem}[section]
% \newtheorem{problem}{Problem}
% \newtheorem{proposition}{Proposition}[section]
% \newtheorem{lemma}{Lemma}[section]
% \newtheorem{corollary}[theorem]{Corollary}
% \newtheorem{example}{Example}[section]
% \newtheorem{definition}[problem]{Definition}
\newcommand{\BEQA}{\begin{eqnarray}}
\newcommand{\EEQA}{\end{eqnarray}}
\newcommand{\define}{\stackrel{\triangle}{=}}
\newcommand*\circled[1]{\tikz[baseline= (char.base)]{
    \node[shape=circle,draw,inner sep=2pt] (char) {#1};}}
\bibliographystyle{IEEEtran}
\providecommand{\mbf}{\mathbf}
\providecommand{\pr}[1]{\ensuremath{\Pr\left(#1\right)}}
\providecommand{\qfunc}[1]{\ensuremath{Q\left(#1\right)}}
\providecommand{\sbrak}[1]{\ensuremath{{}\left[#1\right]}}
\providecommand{\lsbrak}[1]{\ensuremath{{}\left[#1\right.]}}
\providecommand{\rsbrak}[1]{\ensuremath{{}\left[#1\right.]}}
\providecommand{\brak}[1]{\ensuremath{\left(#1\right)}}
\providecommand{\lbrak}[1]{\ensuremath{\left(#1\right.)}
\providecommand{\rbrak}[1]{\ensuremath{\left[#1\right.]}}}
\providecommand{\cbrak}[1]{\ensuremath{\left\{#1\right\}}}
\providecommand{\lcbrak}[1]{\ensuremath{\left\{#1\right.}}
\providecommand{\rcbrak}[1]{\ensuremath{\left.#1\right\}}}
\theoremstyle{remark}
\newtheorem{rem}{Remark}
\newcommand{\sgn}{\mathop{\mathrm{sgn}}}
\providecommand{\abs}[1]{\left\vert#1\right\vert}
\providecommand{\res}[1]{\Res\displaylimits_{#1}} 
\providecommand{\norm}[1]{\left\lVert#1\right\rVert}
\providecommand{\mtx}[1]{\mathbf{#1}}
\providecommand{\mean}[1]{E\left[ #1 \right]}
\providecommand{\fourier}{\overset{\mathcal{F}}{ \rightleftharpoons}}
\providecommand{\system}{\overset{\mathcal{H}}{ \longleftrightarrow}}
% \newcommand{\solution}{\noindent \textbf{Solution: }}
\newcommand{\cosec}{\,\text{cosec}\,}
\newcommand*{\permcomb}[4][0mu]{{{}^{#3}\mkern#1#2_{#4}}}
\newcommand*{\perm}[1][-3mu]{\permcomb[#1]{P}}
\newcommand*{\comb}[1][-1mu]{\permcomb[#1]{C}}
\renewcommand{\thetable}{\arabic{table}} 
\providecommand{\dec}[2]{\ensuremath{\overset{#1}{\underset{#2}{\gtrless}}}}
\newcommand{\myvec}[1]{\ensuremath{\begin{pmatrix}#1\end{pmatrix}}}
\newcommand{\mydet}[1]{\ensuremath{\begin{vmatrix}#1\end{vmatrix}}}
\numberwithin{equation}{section}
\numberwithin{figure}{section}
\numberwithin{table}{section}
\makeatletter
\@addtoreset{figure}{problem}
\makeatother
\let\StandardTheFigure\thefigure{}
\let\vec\mathbf{}
%\renewcommand{\thefigure}{\theproblem}
\title{Assignment 6 Papoulis-Pillai Chapter 4 Example 4.34}
\author{Aayush Prabhu (AI21BTECH11002)}
\date{\today}
\logo{\large \LaTeX}
\begin{frame}
    \titlepage 
  \end{frame}
  \logo{}
  \begin{frame}{Outline}
    \tableofcontents
  \end{frame}
       \section{Question}
       \begin{frame}{Question}
       \textbf{Question:} We place at random 200 points in the interval $(0, 100)$. Find the probability that in the 
       interval (0, 2) there will be one and only one point $(a)$ exactly and $(b)$ using the Poisson approximation. \\
        \end{frame}
        \section{Solution}
        \begin{frame}{Solution}
       \textbf{Solution:} Probability of a point landing in the interval (0,2) is 0.02. i.e $p=0.02$\\
       Probability of $k$ points from 200 points landing in the interval (0,2) is 
       \begin{align}
        &p_k = \comb{n}{k}p^k(1-p)^{n-k}
       \end{align}
       p=0.02, k=1, n=200.\\
       (a) For exact probability
       \begin{align}
       & p_1 = \comb{n}{1}p(1-p)^{n-1}\\
        &p_1 = n\times p\times(1-p)^{n-1}\\
        &p_1 = 200\times 0.02\times(0.98)^{199}\\
        &\therefore p_1 = 0.0718
       \end{align}  
        \end{frame}
        \begin{frame}
       (b) For Poisson approximation, let $\lambda=n\times p$.\\
       in Poisson approximation we take limit $n$ tends to $\infty$ and $p$ tends to $0$. $\lambda$ is a constant.\\
       Substituting value of $p=\dfrac{\lambda}{n}$\\ 
       \begin{align}
           &p_k = \comb{n}{k}p^k(1-p)^{n-k}\\
           &\implies p_k = \dfrac{n!}{(n-k)!k!}\dfrac{\lambda^k}{n^k}(1-\dfrac{\lambda}{n})^n(1-\dfrac{\lambda}{n})^{-k}\\
         & p_k = \dfrac{n(n-1)(n-2)....(n-k+1)}{k!}\dfrac{\lambda^k}{n^k}(1-\dfrac{\lambda}{n})^n(1-\dfrac{\lambda}{n})^{-k}\\
          &p_k = \dfrac{n}{n}\dfrac{n-1}{n}\dfrac{n-2}{n}.....\dfrac{n-k+1}{n}\dfrac{\lambda^k}{k!}(1-\dfrac{\lambda}{n})^n(1-\dfrac{\lambda}{n})^{-k}\\
          &\lim_{n \to \infty} p_k = 1.1.1.....1.\dfrac{\lambda^k}{k!}(1-\dfrac{\lambda}{n})^n.1\\
       \end{align}
      \end{frame}
      \begin{frame}
       \begin{align}
          &\lim_{n \to \infty} (1-\dfrac{\lambda}{n})^n = e^{-\lambda}\\
          &\therefore \lim_{n \to \infty} p_k=\dfrac{\lambda^k}{k!}e^{-\lambda}\\
          &\therefore \lim_{n \to \infty} p_1=\lambda e^{-\lambda}\\
         & \lambda=n\times p \implies \lambda = 0.02\times200 = 4\\
        &\therefore  \lim_{n \to \infty} p_1= 4e^{-4}= 0.0732
        \end{align}
        Value of $p_1$ exactly is $0.0718$ and by Poisson approximation it is $0.0732$
        \end{frame}
\end{document}